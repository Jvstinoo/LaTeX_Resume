%%%%%%%%%%%%%%%%%%%%%%%%%%%%%%%%%%%%%%%%%
% Important note:
% This template requires the resume.cls file to be in the same directory as the
% .tex file. The resume.cls file provides the resume style used for structuring the
% document.
%
%%%%%%%%%%%%%%%%%%%%%%%%%%%%%%%%%%%%%%%%%

%----------------------------------------------------------------------------------------
%	PACKAGES AND OTHER DOCUMENT CONFIGURATIONS
%----------------------------------------------------------------------------------------

\documentclass{resume} % Use the custom resume.cls style

\usepackage[left=0.75in,top=0.3in,right=0.75in,bottom=0.6in]{geometry} % Document margins
\usepackage{hyperref}
\usepackage{xcolor}
\usepackage[normalem]{ulem}
\newcommand{\tab}[1]{\hspace{.2667\textwidth}\rlap{#1}}
\newcommand{\itab}[1]{\hspace{0em}\rlap{#1}}
\name{Justin Balbuena}
\address{Camuy, PR \textbullet\ (787) 210 - 5985}
\address{justinoscar30@gmail.com \textbullet\ 
\href{https://www.linkedin.com/in/jbalbuena}{\color{blue}linkedin.com/in/jbalbuena} \textbullet\
\href{https://www.github.com/Jvstinoo}{\color{blue}github.com/Jvstinoo}}

\begin{document}
%----------------------------------------------------------------------------------------
%	EDUCATION SECTION
%----------------------------------------------------------------------------------------

\begin{rSection}{Education}
%--copy and paste this region  if you need more--
{\bf University of Puerto Rico - Mayagüez} \hfill {\em Aug 2021 - May 2026} 
\\ Bachelor of Science, Software Engineering\hfill { GPA: 3.44 }
\\ {\bf Enrolled Courses:} Intro to SWE, Algorithm Design \verb|&| Analysis
\\
{\bf Completed Courses:} Data Structures (Java), Fundamentals of Computing (Discrete Mathematics), Advanced Programming (C++), Introduction to Computer Programming I (Python)
\end{rSection}
%----------------------------------------------------------------------------------------
%	EXPERIENCE SECTION
%----------------------------------------------------------------------------------------
\begin{rSection}{Experience}
%--copy and paste this region  if you need more--
{\bf Google, Inc.} \hfill {Irvine, CA}
\\
\textit{STEP Intern} \hfill {\em May 2023, Aug 2023}
\vspace{-0.2\baselineskip} % Adjust the space here
\begin{itemize}[noitemsep]
    \item Developed a debugging tool with the ability of processing hundred billion events in a pipeline which resulted in saving thousands of hours of engineering time. This was done as part of the Google Ads Attribution team and used C++ and an internal MapReduce library.
    \item Developed a tool which could generate billions of events in the form of Protocol Buffers for testing purposes. This was done using C++, MapReduce library, and an internal file system.
    \item Went through the process of ramping up, learning new technologies and frameworks which would be used for the main project. 
\end{itemize}

{\bf Google, Inc.} \hfill {Sunnyvale, CA}
\\
\textit{STEP Intern} \hfill {\em May 2022, Aug 2022}
\vspace{-0.2\baselineskip} % Adjust the space here
\begin{itemize}[noitemsep]
    \item Created an internal API which allows users to Grant/Revoke ACLs for Annotation Templates for Core Data’s Datahub team.
    \item Completed entire development process, including writing design docs, implementation, going through design reviews, and launching the API for internal use.
    \item Developed these APIs using C++, Protocol Buffers, and Remote Procedure Calls (RPC).
\end{itemize}
%--copy and paste this region  if you need more--
\end{rSection}
%--------------------------------------------------------------------------------
%    PROJECTS
%-----------------------------------------------------------------------------------------------
\begin{rSection}{Projects}
%--copy and paste this region  if you need more--
{\bf Huffman Coding Tree} \hfill {\em April 2023}\\
Implemented Huffman Encoding Algorithm to encode strings in order to save space. Project done as part of the Data Structures class with the use of Java.
%--copy and paste this region  if you need more--
\end{rSection}
%--------------------------------------------------------------------------------
%    ACTIVITIES
%-----------------------------------------------------------------------------------------------
\begin{rSection}{Academic Activities}
%--copy and paste this region  if you need more--
{\bf Computing Alliance of Hispanic-Serving Institutes (CAHSI)} \hfill {Mayagüez, PR} 
\\
\textit{Webmaster} \hfill {\em Aug 2022 - Aug 2023}
\vspace{-0.2\baselineskip} % Adjust the space here
\begin{itemize}[noitemsep]
    \item Design and develop any Website CAHSI needs for their different efforts such as: Hackathons, Lab, and Research. The tools used were: Figma (Design), WordPress (Development) 
\end{itemize}

{\bf Advanced Programming Lab } \hfill {Mayagüez, PR}
\\
\textit{Mentor} \hfill {\em Aug 2022 - Present}
\vspace{-0.2\baselineskip} % Adjust the space here
\begin{itemize}[noitemsep]
    \item Answer any questions students would have during lab by explaining Object Oriented Programming concepts, C++ syntax, Visual Studio Code errors, Git, GitHub, and others.
    \item Tested the lab beforehand in search for any bugs that the problems might have. 
\end{itemize}
%--copy and paste this region  if you need more--
\end{rSection}
%----------------------------------------------------------------------------------------
%	SKILLS SECTION
%----------------------------------------------------------------------------------------
\begin{rSection}{Skills}
{\bf Software Development:} Python, C++, Java, JavaScript, HTML 5, CSS, Linux, Git

{\bf Languages: }Spanish (Native: Advanced), English (Fluid: Proficient)
\end{rSection}
\end{document}----------------------------


